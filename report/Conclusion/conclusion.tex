

\section{Conclusion - Henrik Falk}
Initially, the goal was to assemble and integrate the parts from the two previous years in the MDH@HOME Butler project. Due to challenges with that system and the big rework that needed to be done. The CAN system needed to be replaced, the design of the base did not allow the robot to travel across cables and, it was; and still is; too wide. A new design was the most viable option with respect to the available time for the project.\\
\indent A big problem, for almost any project, is the delivery time of the parts that are needed. We had selected parts and spent hours researching that they truly would work but in the end we had to order off-the-shelf parts. In a lot of cases, manufacturing our parts were an option. This is very convenient but at the same time a inconvenient for the persons that want to develop on the platform after you have stopped developing. Things will brake and if the part is not easily purchasable and replaceable, then a lot of time will go to waste.\\
\indent The Husqvarna Research Platform is a versatile base but could be quite limiting since it is "semi-open". Husqvarna needs to show good will towards the applied researching community to really make the platform flourish with its current potential.\\
\indent  showing the wrong values When buying cheap hardware it is not only the quality of the hardware but also the datasheets that is degraded. This was a massive issue for implementing the motors and motor controllers. This was not the case with the Cortex-M4F and NVIDIA Jetson TX2 which was successfully implemented and functional. The Cortex-M4F motor control software showed some boundaries which is unclear why they are there.\\
\indent The new CAN system was achieved as intended but in the manufacturing of the CAN transceiver there were some milling problems. These were corrected. Also, the current implementation of CAN in ROS is only able to write but not receive.\\
\indent The object recognition of a coffee cup is working but is limited. It is viable but needs a bigger data-set to mature.\\
\indent This became a big project from a compartmentalised assignment but the budget did not scale accordingly. With the combined effort of the project team, the project spending broke the budget ceiling with 2300 SEK.\\
\indent Cross-developing in the project course presented itself as fruitful. A lot of the implementations from the UNICORN project could be directly applied on the Butler. The heavy focus on ROS our contribution to the researcher and fellow students at the university is this system and knowledge base that we have created over the past twenty weeks.\\
\indent This years project explored a different path. The objectives were not completely fulfilled but with the help of this documentation it is achievable.



% Initial challenges - too wide, collab, motivation
% Parts - delivery
% This became a big project from a initial compartmentalised assignment of assembly and rework to a complete robot.....

% Manufacturing
% Budget




% from report
    % husqvarna - limitations, semiclosed
    % Motors - datasheet and reality two different things, cheap motors
    % Motor Controller - same thing again with the sheets
    % Cortex-M4F - implemented and functional
    % Jetson TX2 - operational working as intended
    % CAN transceiver - milling process error but works
    % PDB - manufactured but not tested with the motors etcetera. Battery specified but not purchased
    % CAN protocol - achieved
    % Motor control M4F - boundaries found but reason unclear
    % ROS CAN - implemented only for write and not read, needed for feedback
    % Object recognition - limited range of detecting mugs, lighting posed a problem.
    
    


% exists industrial parts for most things but it has a long delivery time.

% Positive - crossdeveloping two projects,  contribution research (can specially) - ROS introduction

% Though this is an applied project we have introduced researchers at Mälardalen University to the application of ROS and its benefits.

% Talk about the parallel platform



