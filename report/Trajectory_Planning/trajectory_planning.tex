\section{Trajectory Planning - Niclas Holmqvist}
\subsection{Platform}
The prestudies done for this topic states that even though analytic inverse kinematic solution would be the best solution it is a too complex problem and would most likely take too long to implement. Instead, initially it was decided to create an inverse kinematics solver which would be easy to implement and use. However, the scope of the project included things like trajectory planning, object avoidance, self collision avoidance, constraints in space and in joints. These things are not fulfilled by only solving the kinematics problem.\\
\indent When looking at how to realise trajectory planning it was found that there are programs which can be powerful tools when controlling robot arms. MoveIt is one of them. It provides a platform that is easy to use when developing a robot. MoveIt already uses ROS which motivated the decision to develop the planner using MoveIt.

\subsection{Unified Robot Description Format}
\indent Designing a robot from scratch takes time and seeing as a manual set up of a MoveIt package was required it was decided a temporary simple URDF file was needed. It was designed so that it resembles the Butler enough so that it would be possible to get a MoveIt package ready and running. The simple URDF consisted of cylinders which were connected by joints in the same way the Butler were going to be using. This URDF was used when setting up the basic configuration of the MoveIt package.\\
\indent The URDF supplied by the mechanics team resembles the Butler robot in more detail with which it was possible to test more things than what was possible with the simple URDF. For example it was possible to test the more in detail described robot model for self collisions. In Appendix \ref{app:urdf} Figure \ref{urdf}, a tree depicts the different frames of the URDF of the robot and how they are connected with the world reference frame as a root. This URDF replaced the simple URDF which already contained the same joint setup and lengths making the transition straightforward.



\subsection{MoveIt}
MoveIt is a robotics software embedded into ROS. Its main component is the node movegroup. This node is setup by configuring peripheral nodes around it. For example the URDF is one parameter which is loaded into the ROS parameter server peripheral. Other parameters that use the ROS param server are joint limits, kinematics, motion planning, perception among others. The parameter server is globally available and makes it easy for tools to inspect the configuration parameters to recognise the current state of the system.\\
\indent Robot Controllers is another peripheral node connected to the movegroup node. This node controls the joints of the robot with a JointTrajectoryAction message. An extension of this node is the roscontrol package setup to interface the hardware used in the robot. All sensors involved in supplying the system with information on how to compute trajectory of the robot are also set up as peripherals to the movegroup node to supply the planner necessary data such as current state information, positions of the joints, point clouds, etc. to avoid obstacles and singularities.\\
\indent One more peripheral of the movegroup node is the user interface. It supplies an easy to use interface which can be setup in C++, Python or used through a graphical motion planning user interface used in Rviz. An interface is setup in C++ which allows the user to set target tool centre point (TCP) together with a pose of the end effector. It is possible to define both the pose and target TCP from scratch with keyboard input as well as giving the arm a command to, for example, move the TCP 0.1m to the left in the cartesian space. Let's call the combination of a pose and a target TCP a goal. When setting a goal the motion planner will calculate a trajectory for the robot arm to reach its goal while checking constraints. The only constraints used by the planner for the Butler robot so far is the joint constraint as well as maximum velocities and accelerations for each joint. This means the Butler robot will plan the trajectory while restricting joints to lie within the specified values of the constraint and respecting speed and acceleration limits. Self collision is also avoided.\\
\indent There are additional constraints ready to be set up which are inbuilt in the motion planner such as orientation of a link to lie within a specified roll pitch yaw limit or limit a point on a link to lie within a specified visibility cone for a sensor.\\
\indent The motion plan result is a message which contain the information to move the arm to the goal with respect to the constraints. As seen in Appendix \ref{app:urdf} Figure \ref{traj_ex}, the trajectory is split up into sub goals with joint space values of each joint, acceleration and velocity of each joint, and each sub goal has a time stamp. Performing the movement according to the message will result in a successful motion.


\subsection{Kinematics solver}
Kinematics are calculated in a plugin infrastructure used by MoveIt. There is a default solver automatically configured by MoveIt but a custom solver was implemented using IKFast. It allows an analytic solver to be configured to compute the kinematics of the robot in an analytic approach.

