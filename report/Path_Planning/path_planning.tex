\section{Path Planning - Josefine Häll}

The path planner was developed together with the path planner for the UNICORN-project that was carried out in the fall of 2017 at Mälardalen University. 

The path planner chosen was a package built in ROS that calculates a path to be taken. There are different configurations possible to control the behaviour of the planner. Some examples of settings to be considered is the footprint of the robot and how close to the obstacles the robot is allowed to be. The path planner contains both a local and a global planner that calculates the path in different ways. The global path calculates the path from the robots position to the goal while the local planner calculates a smaller path as the robot is moving. To have two different planners is useful in several aspects. In the case of Charlie, the local planner takes care of the small movements that the robot takes, since the robot is not exact enough to follow the global path directly. In other cases it is useful if big areas are to be covered. In that case a global path can be calculated on the big scale with a lower resolution while the local one is calculated with a higher resolution of the map \cite{shamsudin2017two}.

\subsection{Global Planner}
The global planner calculates the path all the way from the start position to the goal. The global planner calculates the path from a global costmap which is created from the map over the environment. There are many configurations that can be considered in the costmap, such as how far away the robot must stay from obstacles in the map.

\subsection{Local Planner}
The local planner calculates the path from the current position and a bit forward. The local costmap is continuously updated and moved through the global costmap. There were also configurations for the local planner to control its behaviour. There were two different local planners tested. The first one was the base\_local\_planner. This planner calculates a trajectory forward which is, among other things, based on how close it will stay to the global path and how far away it will be to obstacles around it\footnote{\url{http://kaiyuzheng.me/documents/navguide.pdf}}. This local planner did not work as well as needed, therefore another one was tested. The second local planner tested was the teb\_local\_planner. Instead of calculating a distance forward and putting the end point close to the global path, the endpoint is always put on the global path and a path from the robots position to the point on the global path is calculated\footnote{\url{http://wiki.ros.org/teb_local_planner}}. 

%The path planner chosen was a package built in ROS that calculates a path to be taken. There are different configurations possible to control the behaviour of the planner. Some examples of settings to be considered is the footprint of the robot and how close to the obstacles the robot is allowed to be. The path planner contains both a local and a global planner that calculates the path in different ways. The global path calculates the path from the robots position to the goal while the local planner calculates a smaller path as the robot is moving. To have two different planners is useful in several aspects. In the case of the UNICORN, the local planner takes care of the small movements that the robot takes, since the robot is not exact enough to follow the global path directly. In other cases it is useful if big areas are to be covered. In that case a global path can be calculated on the big scale with a lower resolution while the local one is calculated with a higher resolution of the map\cite{shamsudin2017two}.

%\subsection{Global Planner}
%The global planner calculates the path all the way from the start position to the goal. The global planner calculates the path from a global costmap which is created from the map over the environment. There are many configurations that can be considered in the costmap, such as how far away the robot must stay from obstacles in the map.

%\subsection{Local Planner}
%The local planner calculates the path from the current position and a bit forward. The local costmap is continually updated and moved through the global costmap. There were configurations also for the local planner to control it's behaviour. There were two different local planners tested. The first one was the base\_local\_planner. This planner calculates a trajectory forward which is, among other things, based on how close it will stay to the global path and how far away it will be to obstacles around it\cite{pathplannertuningguide}. This local planner did not work as well as needed, therefore another one was tested. The second local planner tested was the teb\_local\_planner and the calculations for that is a bit different. Instead of calculating a distance forward and putting the end point close to the global path, the endpoint is always put on the global path and a path from the robots position to the point on the global path is calculated\cite{teblocalplannerpackagesite}. 